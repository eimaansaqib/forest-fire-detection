\section{Conclusion}

In conclusion, our study presents a comprehensive evaluation of deep learning segmentation models for forest fire detection tasks. By exploring the impact of various parameters and architectures, we have  highlighted the potential of transformer-based architectures for improving these tasks. Although our study was limited by the moderately sized and biased dataset (FLAME), our findings have significant implications for the fields of forest fire detection and computer vision. 

The potential applications of our findings include improving early detection and response to forest fires, which can ultimately help to minimize the damage and loss of life caused by these disasters. Additionally, our study contributes to the broader field of computer vision by providing further evidence of the effectiveness of transformer-based architectures in image segmentation tasks. 

Future research could focus on addressing the limitations of our study by utilizing larger and more diverse datasets, examining the impact of different hyperparameters and data augmentation techniques on model performance, and investigating the potential of semi-supervised learning techniques for forest fire detection tasks. The exploration of transformer-based architectures in other computer vision tasks could also be a promising avenue for future research.
